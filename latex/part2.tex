\section{Classification \& Bio-geophysical Parameter Retrieval}

\subsection{Can we predict where there is chlorophyll through classification?}

We will use \textit{K-Means clustering} to classify the data. K-means clustering is a 
unsupervised learning algorithm that can be used to classify and cluster data into $k$ 
different clusters. The data points are adjusted iteratively until all points are 
associated with the nearest cluster. We want to cluster each observation (pixels, with $n$ spectral channels) into a specific 
cluster (environment class, ie. deep water, shallow water, vegetation). 


\begin{figure}
    \centering
    \includegraphics[width=\textwidth]{../fig/kmean/kmean_8.png}
    \caption{K-mean clusters of the image}
    \label{fig:kmean}
\end{figure}

As we can see from \cref{fig:point_spectra}, we know that those three different points 
have distinctly different spectra, thus it should be possible to classify them accordingly. 
The results of a K-mean clustering, run with Spectral Python's \textit{kmeans} function \cite{website:spectral}, 
can be seen in \cref{fig:kmean}. We clearly see different classes for water, land, and vegetation, the latter 
containing lots of chlorophyll. We also see a very distinct class along the coast on the upper part 
of the image. This may very well be a collection of chlorophyll, but it might also just be shallow water, or 
more likely a combination of both. 

\subsection{How well can we directly estimate the chlorophyll content?}

We use the NASA OBPG algorithm, defined in equation 4 in the assignment 
\cite{assignment}, as well as the parameters given there, to try to 
visualize the chlorophyll contents. Using the closest available wavelengths 
in the dataset, $\lambda_{green} = 553$ (i = 26) and $\lambda_{blue} = [444, 490, 507]$ 
(i = [7, 15, 18]). The results can be seen in \cref{fig:obpg}. We can clearly 
see high concentrations on the north west coast, same place as in \cref{fig:kmean}, 
but now we also see quite a bit on the southern coast as well. Thus it seems that this 
algorithm performs better than the k-means clustering. 

\begin{figure}
    \centering
    \includegraphics[width=\textwidth]{../fig/2b_nasa.png}
    \caption{Results from the NASA OBPG algorithm, showing chlorophyll concentrations}
    \label{fig:obpg}
\end{figure}

\subsection{How can we estimate the reflectance from the surface of the ocean?}

The data in the HICO dataset actually contains the measurement of radiance exiting the top 
of the atmosphere, and not the radiance of the water directly. Therefore we must recover the 
radiance $R_{rs}$ of the water from the top of atmosphere (TOA) measurements. This is done with 
the empirical line (ELM) method, as described in \cite{assignment}, on the form \cref{eq:R_rs}. 
Where $a$ and $b$ are terms that model the absorption of light in the atmosphere, which is to 
be estimated, and $L$ is the measured value in the HICO dataset. 

\begin{equation}
    \label{eq:R_rs}
    R_{rs}(\lambda) = \frac{L(\lambda) - b(\lambda)}{a(\lambda)}
\end{equation}

\begin{figure}
    \centering
    \includegraphics[width=\textwidth]{../fig/pseudo_rgb_corrected.png}
    \caption{Pseudo RGB image of the atmosphere corrected image}
    \label{fig:RGB_corrected}
\end{figure}

After performing the atmospheric correction, the resulting image is shown in 
\cref{fig:RGB_corrected}. It is very difficult to see a clear difference between 
the two by only looking at the image, but if we look at the pixel values for the red, 
green and blue channels separately, we see that the blue color channel is only about 2\% 
of the original, uncorrected pixel value, while the red and blue channels are about 9-10\% 
of their original values. Thus it would seem that the atmospheric correction removes some of 
the blue color of the image. 

\subsection{Compute chlorophyll concentration using atmosphere-corrected data}

\subsection{Classify land versus water}

\subsection{Other bio-geophysical parameters}

\subsection{Alternative atmospheric correction methods}