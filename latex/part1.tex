\section{Getting familiar with the data}

\subsection{Finding the spectral resolution}

To find the spectral resoulution of the dataset, we load the \textit{hico\_wl} array, which contains the
wavelength corresponding to band \textit{i}. We loop through the array and compare each wavelength
\textit{i} with the the previous wavelenght \textit{i-1} and we find that the average distance between
the wavelengths is $5.728 nm$, which seems to be constant between all wavelengths. 

\subsection{Relation to human color perception}

\begin{figure}
    \centering
    \includegraphics[width=\textwidth]{../fig/human_spectrum.png}
    \caption{Graph for the human color sensitivity curves, according to Wikipedia \cite{website:wiki_spectral}}
    \label{fig:human_spectra}
\end{figure}

The color sensitivity of the human eye is shown in \cref{fig:human_spectra}. As we can 
see, blue color has a peak around $450 nm$ (\textit{S}-curve), green peaks at $550 nm$ 
(\textit{M}-curve), and red at $600 nm$ (\textit{L}-curve). 

\subsection{Create a pseudo RGB image from the hyperspectral bands}

From the \textit{hico\_wl} array, we find that Blue (450nm) is located at index \textit{i = 8}, 
green (550nm) at \textit{i = 25}, and finally red (600nm) at \textit{i = 34}. We combine these indices from 
the HICO dataset and show it as an image to create a pseudo RGB image, shown in \cref{fig:pseudo_rgb}.


\begin{figure}
    \centering
    \includegraphics[width=\textwidth]{../fig/pseudo_rgb.png}
    \caption{Pseudo RGB image, showing R (600nm), G (550nm), B (450nm)}
    \label{fig:pseudo_rgb}
\end{figure}

\subsection{Representative spectra for selected points}

We want to look at the representative spectra of the points (20,20), (100,70) and (400,30), which is in deep water, shallow water 
and vegetation respectively.
As we can see in \cref{fig:point_spectra}, we see that there is a clear difference in the spectra between water and vegetation. 
Both have amplitude peaks at the lower end of the spectra and then drop off in power as the wavelength increases. Vegetation however increases
again in power at a wavelength of around 700nm, while the water is still decreasing. 
The findings here seem to agree to the findings of Lucke et al \cite{Lucke:11} as the general shape of the curves matches those of figure 12 in 
that report. 

\todo{Fix image, fix question}

\begin{figure}
    \centering
    \includegraphics[width=\textwidth]{../fig/pseudo_rgb_points.png}
    \caption{Representative spectra of spesific points}
    \label{fig:point_spectra}
\end{figure}

