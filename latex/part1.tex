\section{Getting familiar with the data}

\subsection{Finding the spectral resolution}

To find the spectral resoulution of the dataset, we load the \textit{hico\_wl} array, which contains the
wavelength corresponding to band \textit{i}. We loop through the array and compare each wavelength
\textit{i} with the the previous wavelenght \textit{i-1} and we find that the average distance between
the wavelengths is $5.728 nm$, which seems to be constant between all wavelengths. 

\subsection{Relation to human color perception}

\begin{figure}
    \centering
    \includegraphics[width=\textwidth]{../fig/human_spectrum.png}
    \caption{Graph for the human color sensitivity curves, according to Wikipedia \cite{website:wiki_spectral}}
    \label{fig:human_spectra}
\end{figure}

The color sensitivity of the human eye is shown in \cref{fig:human_spectra}. As we can 
see, blue color has a peak around $445 nm$ (\textit{S}-curve), green peaks at $535 nm$ 
(\textit{M}-curve), and red at $575 nm$ (\textit{L}-curve). 

\subsection{Create a pseudo RGB image from the hyperspectral bands}

From the \textit{hico\_wl} array, we find that Blue (445nm) is located at index \textit{i = 7}, 
green (535nm) at \textit{i = 23}, and finally red (575nm) at \textit{i = 30}. We combine these indices from 
the HICO dataset and show it as an image to create a pseudo RGB image, shown in \cref{fig:pseudo_rgb}.

\todo{Fix image}

\begin{figure}
    \centering
    \includegraphics[width=\textwidth]{../fig/pseudo_rgb.png}
    \caption{Pseudo RGB image}
    \label{fig:pseudo_rgb}
\end{figure}

\subsection{Representative spectra for selected points}
